
\documentclass{article}
\usepackage{showkeys}
\usepackage{enumerate}

% Title page information
\title{
CODE OF CONDUCT GOVERNING\\
\textbf{COMPUTER SCIENCE HOUSE'S}\\
COMPUTER SYSTEMS AND FACILITIES}
\author{Computer Science House RTPs}

% Last Modified Date
\newcommand{\datechanged}{Proposed: \today}
\date{\datechanged}

% Fix margins
\setlength{\evensidemargin}{0in}
\setlength{\oddsidemargin}{0in}
\setlength{\textwidth}{6.5in}
\setlength{\topmargin}{0in}
\setlength{\textheight}{8.5in}

\pagestyle{myheadings}
\markright{{\rm CSH Code of Conduct \hfill \datechanged \hfill Page }}

\begin{document}
\maketitle

Computer facilities at Computer Science House are shared by the entire membership of CSH and are connected to various networks, which in turn connect us to the facilities of RIT and the rest of the world. In recognition of this and the fact that CSH also gains it's right to exist through the blessings of RIT, the following guidelines have been established for all members of CSH. Please read the following rules and sign your name at the bottom to indicate that you have read, understand, and agree to abide by all of the rules. Furthermore, by signing this sheet you acknowledge that you have read and will also follow I.T.S.'s Code of Conduct.

\begin{enumerate}[I]
\item You may not use any CSH computer system toward any commercial or business ends. CSH and RIT are non-profit organizations, and any such use of our systems could potentially jeopardize that standing.
\item You must not reveal your password to anyone else for any reason.
\item You are responsible for your account and your computer, should you attach it to the CSH network. Any damage or abuse discovered to have come from your account, with or without your knowledge, will be accounted to you and the perpetrator. You will be held equally responsible and will be dealt with accordingly.
\item You may not change, copy, delete, read, or otherwise access files and resources which you do not own (regardless of the actual permissions) without the expressed permission of the owner.
\item You may not prevent other authorized users from accessing and using the facilities of CSH.
\item You may not waste valuable server and network services. This includes such activities as running wasteful or repetitive jobs, intentionally driving up the load average, or needlessly consuming valuable system resources.
\item You may not intentionally bypass security mechanisms or attempt to circumvent data protection schemes. This includes, but is not limited to, using other people’s accounts and attempting to gain unauthorized access to privileged accounts.
\item You may not harass others by sending annoying, obscene, libelous or threatening messages. This includes messages sent through mail, news and posts to the information system.
\item You may not intentionally misrepresent our machines or yourself over the network.
\item You may not conduct illegal activity through CSH computing resources, including the network. This includes but is not limited to DoS attacks, theft, fraud, and downloading software, music or other multi-media that you do not own.
\item You may not take advantage of open terminals/ports which still have someone logged in. If such a connection is found, you should log the person off and send mail to them indicating that they left themselves logged in on a public terminal. This especially applies to console connections that never terminate properly.
\item You will refrain from handling hardware with which you are not familiar. If you don't know what it is or does, don't touch or move it.
\item Under no circumstances should non-members be given access to CSH facilities without another member being present at all times.
\item Under no circumstances should copyrighted source code leave our site. Examples of such are Ultrix, original UNIX, or Inferno. Should you ever be given access to a privileged account with access to these sources, you are forbidden to distribute them to any other site or installation.
\item You will note that having a computer account is a privilege, NOT a right, and abuse of any account will not be tolerated.
\item All computer accounts and the files contained within them are the property of RIT. Therefore, RTP’s and the Executive Board have the right to review and delete accounts or their contents.
\item All system usage and network traffic is susceptible to logging and monitoring.
\item This Code of Conduct is an extension of the ITS Code of Conduct, and as such, implicitly contains all the policies described within the official ITS Code of Conduct. Note that any future modifications of the ITS Code of Conduct WILL BE AUTOMATICALLY INCLUDED INTO THIS AGREEMENT.
\end{enumerate}

There is no time limit or statute of limitations to this agreement; the rules of this document (and the documents it refers to) are binding throughout your membership in Computer Science House (including, but not limited to, post-RIT membership status such as Alumni, Inactive Member, and off floor member).

The active System Administrators (RTP’s) and the Executive board shall have full power to police and enforce these regulations. Violators may have their accounts deactivated and or membership revoked. In addition, they may face RIT judicial boards and criminal charges. Although Computer Science House will handle most violations of these rules internally, severe violations may be handled by RIT directly. Computer Science House is obligated to notify ITS of any Code of Conduct violations.


\noindent
\begin{tabular}{ll}
\\[8ex]
\makebox[2.5in]{\hrulefill} & \makebox[2.5in]{\hrulefill}\\
Print Name & CSH Username\\[8ex]
\makebox[2.5in]{\hrulefill} & \makebox[2.5in]{\hrulefill}\\
RIT Username & RIT ID Number\\[8ex]
\makebox[2.5in]{\hrulefill} & \makebox[2.5in]{\hrulefill}\\
Signature & Date\\
\end{tabular}

\end{document}
